\chapter{Applications}
\label{chap:applications}


To illustrate the use of the graphical user interface, two different physical
applications are shown in the desktop and the web frontend, respectively. From
the physics point of view, the example in desktop version focuses more on the
density of states and the visualization of spin contributions, while the dataset
in web frontend focuses more on the band structure $E(k)$ and the visualization
of defect states in supercells.

\section{Web Frontend: $MoSe_2$ Crystal}

Figure \textbf{TODO} shows the visualization of a band structure calculation of
a 3 dimensional Molybdenum diselenide ($MoSe_2$ bulk) crystal using the default
settings of the GUI. Even with the default settings the band structure plots
clearly indicate, that $MoSe_2$ is a semiconductor, since there are no states at
the Fermi surface. Because the minimum of the conduction band is located at an
other $k$ as the maximum of the valence band, the plot shows that $MoSe_2$ has
an indirect band gap. This indicates that for the transition with the smallest energy difference between valence and the conduction band, both, energy and momentum have to change.

In contrast to the 3 dimensional extended $MoSe_2$ crystal, a $MoSe_2$ monolayer
(see Fig. \textbf{TODO}) has a direct band gap but is still a semiconductor.
Further more, the $MoSe_2$ monolayer has a defect atom in every 9th unit cell
and the DFT computation is therefore done in a $3 \times 3$ supercell to restore
periodicity. This is the reason for the much greater number of states in the
band structure plot. 

Since the Brillouin zone of the supercell is smaller than the Brillouin zone of
the same crystal without the defect, the supercell Brillouin zone is unfolded
to the same size as the bBrillouin zone of the unperturbed lattice. To account
for the fact that the defect is only present in every 9th cell and its relative importance for the spectrum is therefore degraded, an unfolding weight is introduced to visualize the relative importance of bands in the unfolded crystal. By default, the unfolding weight is used by our visualization tool but it can be gradually turned off in order to highlight the impact of defect states.
This is shown in figure \textbf{TODO} to even better visualize the defect state,
it would also be possible to select the atom group belonging the defect atoms
only. In this case the analysis with reduced band unfolding shows, that there
are many more direct band gaps origination from the defect state.




\section{Desktop Frontend: $Co$ Crystal}

\subsection{Theory}


% bänder isonlieren schwierig....fleur update--> label der E eigenvlas

%$m^{*}$ and $v_{G}(\vec{k})$
\section{Derived Quantities : Differentiation}
The kind of datasets handled in the scope of the project did not lend themselves
readily to applications of automatic differentiation techniques. Nevertheless,
it is possible to derive meaningful physical quantities from the band structure
using numerical differentiation techniques.

The effective mass $m^{*}$ represents the mass, that an electron appears to have due to the inter-atomic forces in the crystal. At every $\vec{k_0}$, where $E(\vec{k})$ has a local extremum, $E(\vec{k})$ can be expanded in a Taylor series with a vanishing first order term $E(k) = E_0 + \frac{\partial E(\vec{k})}{\partial \vec{k}} \cdot (\vec{k}-\vec{k}_0)^2$. Comparing this to the dispersion relation of a free electron $E(\vec{k})_{free} = \frac{\hbar^2 \vec{k}^2}{2 m_e}$ yields the general definition 

\begin{equation}
    m^{*}\big|_{i,j} = \hbar^2  \left(\frac{\partial^2E(\vec{k})}{\partial k_i \partial k_j}\right)^{-1}
\end{equation}

Since $\frac{\partial^2E(k)}{\partial \vec{k}_i \partial \vec{k}_j}$ depends on the direction of the gradient, $m^{*}\big|_{i,j}$ is a tensor. Since the band structure files only contain a discrete sampled path in the Brillouin zone, only the derivatives that correspond to the direction from one high-symmetry point to the next can be computed. We are only interested in the diagonal terms of $m^{*}\big|_{i,j}$.

%maybe something about the directional dependency --> m is tensor...

Further more, the group that correspond to excitations of the eigenstates can be defined: 

\begin{equation}
    v_{G}(\vec{k})\big|_i = \frac{1}{\hbar}\frac{\partial E(\vec{k})}{\partial \vec{k_i}}
\end{equation}

where the group velocity $v_{G}(\vec{k})$ at the Fermi energy $E = E_F$ is called Fermi velocity.



\subsection{Differentiation}

Since the k-mesh in DFT calculations is potentially very sparse, low order finite difference schemes are not expected to work well.

Alternatively, one way to exploit all data points efficiently is to use fast Fourier transform methods, which are equivalent to the derivation of a truncated Fourier series. This method is expected be well-suited for the problem since the graph of the band structure $(k, E(|k|))$ with $k \in \{-\Gamma,..., H,..., \Gamma\}$ is periodic, where $\Gamma$ and $H$ are arbitrary representatives of the high symmetry points. This periodicity in the reciprocal space is a direct consequence of the periodicity of the crystal.


In Fourier space, spatial derivatives transform into multiplications, which can easily be shown by partial integration.
\begin{equation}
    f^{(n)}(x) = \mathcal{F}^{-1}\left((ik)^n\mathcal{F}(f(x))\right)
\end{equation}
 
To test the FFT differentiation method, a band was selected that did not have intersections with other bands within the interval between two high symmetry points and a stationary point at the high symmetry points. Then a resolution study was done to investigate the impact of the number of points within the interval.

The result with the comparison between the FFT and a first-order central
difference approximation of the second derivative (FD) is shown in Fig. \textbf{TODO}, where
various grid spacings are compared to a derivative that is almost fully converged.

The comparison indicates that for small N, the error of the FFT method is significantly smaller than the error of the FD derivative. This is especially apparent in the region of the high symmetry points, where the error of the differentiation method must be small in order to get good approximations for $m^{*}$, which in turn only makes sense at the high symmetry points.

When using more points, the difference between both approximation schemes is not striking. It is interesting to note, that the FFT method does not work anymore in the limit of extremely many points. In this case it is dominated by Gibbs oscillations. These are most likely caused by small discontinuities in $E(k)$, due to basis changes inside the DFT computation.


At the current state the question remains to be answered if the computation of $m^{*}$ and $v_g$ is
useful. The number of bands, which the computation can be applied to is
extremely limited. Since the bands in the data files are not labeled according to their corresponding eigenfunction but sorted by value, there are discontinuities at each point, where two bands intersect. 













 %   \item Effective mass, that an electon in a crystal appears to have compared to a free %electron (due to interactions in the solid)
%    $m^{*} = \hbar^2  \left(\frac{\partial^2E(k)}{\partial k^2}\right)^{-1}$
%    \
%    \item Group velocity:
%    $v_{G}(\vec{k}) = \frac{1}{\hbar}\frac{\partial E(\vec{k})}{\partial \vec{k}}$%
%
%    
%    \item Problem: sparse k-Point mesh, but periodic band structure
%    \item Idea: Using FFT to compute accurate derivates:
%    \newline $\Leftrightarrow$ Differentiate a finite Fourier series
%    \newline $f^{(n)}(x) = \mathcal{F}^{-1}\left((ik)^n\mathcal{F}(f(x))\right)$


%%% Local Variables:
%%% mode: latex
%%% TeX-master: "../report"
%%% End:
